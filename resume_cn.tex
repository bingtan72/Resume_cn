\documentclass{resume_cover_letter}
\usepackage{setspace}
\usepackage{xeCJK}
\usepackage{ebgaramond}

%\defaultfontfeatures{Mapping=tex-text}  % 启用tex风格字符
%\setmainfont{Times New Roman}           % 英文缺省字体
%\setsansfont{Arial}                     % 英文无衬线字体
%\setmonofont{Courier New}               % 英文打字机(等宽)字体


%\setmainfont[Ligatures=TeX,SmallCapsFeatures={Letters=SmallCaps,LetterSpace=5}]{EB Garamond}
%\setsansfont{Lato}


\setCJKmainfont[BoldFont={方正兰亭中黑_GBK}]{Adobe Kaiti Std}
\setCJKsansfont{SimHei}
\setCJKmonofont{[FangSong_GB2312]}

%other optional fonts: Adobe Heiti Std/Adobe Song Std/Adobe Fangsong Std 
%\setCJKmainfont[BoldFont={Adobe Heiti Std},ItalicFont={Adobe Kaiti Std}]{Adobe Song Std}
%\setCJKsansfont{Adobe Heiti Std}
%\setCJKmonofont{Adobe Fangsong Std}





% infos
\firstname{谭}
\familyname{兵}
\title{四川/2021届/中共党员
	\hspace{200pt}求职意向:大学教师}
\address{}{\faHome~四川--成都}
\mobile{183 2807 1036}
\email{bingtan72@gmail.com}
%\social[github]{Martin1205}
%\social[weibo]{TiancaiMartin}
%\social[twitter]{TiancaiMartin}
\social[qq]{642314145 (Attitude)}
\social[weixin]{Attitude199572}
%\social[facebook]{朱震宇}
\homepage{bingtan.me}

%\extrainfo{\faCertificate 驾驶执照(C1)}
\photo[90pt][0pt]{img/bingtan.jpg}	

\begin{document}

\maketitle

\vspace*{-17mm}
\section{\faGraduationCap~教育}

\tlcventry{2018/9}{2021/7}{电子科技大学}{数学}{硕士}{}{导师:李颂孝,秦小龙}
\tlcventry{2014/9}{2018/7}{西南石油大学}{数学与应用数学}{本科}{}{毕业论文:分裂 Bregman 方法在图像处理中的一些应用}


\section{\faBriefcase~经历}
\begin{onehalfspacing}

\end{onehalfspacing}
%\section{\faGavel~项目}

%\tldatecventry{2016}{Rock Paper Scissors Lizard Spock %\faHandRockO~\faHandPaperO~\faHandScissorsO~\faHandLizardO~\faHandSpockO}{University Institute of %Technology}{Montpellier, France}{\url{www.gaelfoppolo.com/projets/pfcls/}}{The project is based on the }


%\section{\faNewspaperO~获奖与表彰}
\section{\faTrophy~获奖}
\tldatecventry{2019/10}{电子科技大学一等学业奖学金}{}{}{}{}
\tldatecventry{2019/7}{电子科技大学基础院优秀共产党员、电子科技大学校优秀研究生干部}{}{}{}{}
\tldatecventry{2019/7}{BOE校园俱乐部 “优秀主席”. 注:京东方科技集团股份有限公司人力资源部.}{}{}{}{}
\tldatecventry{2018/9}{电子科技大学二等学业奖学金}{}{}{}{}
\tldatecventry{2018/6}{被聘请为西南石油大学 “蓝鲑鱼计划” 成员. 注:鲑鱼洄游,感恩母校,分享青春}{}{}{}{}
\tldatecventry{2018/6}{西南石油大学2018届本科优秀毕业论文 (2\%)}{}{}{}{}
\tldatecventry{2018/3}{西南石油大学2018届校级优秀毕业生}{}{}{}{}
\tlcventry{2015}{2017}{西南石油大学第八届、第九届、第十届 “睿智杯” 大学生数学竞赛一等奖}{}{}{}{}
\tlcventry{2015}{2017}{西南石油大学第八届、第九届、第十届 “睿智杯” 大学生数学竞赛一等奖}{}{}{}{}
\tlcventry{2014}{2018}{2 次校级三好学生 (4\%), 1 次校级优秀学生干部}{}{}{}{}
\tldatecventry{2017/11}{校级重点项目《二孩政策对我国人口发展的影响》结题验收一等奖}{}{}{}{}
\tldatecventry{2016/11}{校级重点项目《基于肤色的人脸识别技术研究》结题验收三等奖}{}{}{}{}
\tldatecventry{2018/5}{西南石油大学优秀团员}{}{}{}{}
\tldatecventry{2017/6}{第五届 “泰迪杯” 全国数据挖掘挑战赛 MATLAB 创新奖 (全国仅两项)}{}{}{}{}
\tldatecventry{2017/12}{全国大学生数学建模竞赛本科组全国一等奖 (1\%)}{}{}{}{}
\tldatecventry{2017/4}{美国大学生数学建模竞赛(MCM/ICM)一等奖}{}{}{}{}
\tldatecventry{2015/12}{全国大学生数学建模竞赛本科组全国二等奖(6\%)}{}{}{}{}



%\cvitem{其他}{\textbf{\textit{Dean's List},校级优秀三好学生,厦门大学一等学业奖学金}}

%\section{\faGears~技能}
%\begin{minipage}{\linewidth}
%  \begin{multicols}{3}
%    \centering{\textbf{编程语言\\}}
%    \vspace*{2mm}
%    %\centering{Python~\textsc{\faStar~\faStar~\faStar~\faStarO~\faStarO}\\}
%    \centering{Matlab~\textsc{\faStar~\faStar~\faStar~\faStarHalfO~\faStarO}\\}
%    %\centering{~~~R~~~\textsc{\faStar~\faStar~\faStar~\faStar~\faStarO}\\}
%    \columnbreak
%    \centering{\textbf{工具\\}}
%    \vspace*{2mm}
%    %\centering{GitHub~~~~~~\textsc{\faStar~\faStar~\faStar~\faStarO~\faStarO}\\}
%    \centering{Office~~\textsc{\faStar~\faStar~\faStar~\faStar~\faStarO}\\}
%    \centering{\LaTeX~~~\textsc{\faStar~\faStar~\faStar~\faStarHalfO~\faStarO}\\}
%    \columnbreak
%    \centering{\textbf{软件\\}}
%    \vspace*{2mm}
%    \centering{SAS~~~\textsc{\faStar~\faStar~\faStar~\faStarHalfO~\faStarO}\\}
%    \centering{SPSS~~\textsc{\faStar~\faStar~\faStar~\faStarO~\faStarO}\\}
%  \end{multicols}
%\end{minipage}

%\section{\faPencil~自我评价}
%%\faHandORight
%\begin{raggedright}
%\faThumbsUp \textbf{通过项目和竞赛学习到了一种方法:理论 \faLongArrowRight 实践 \faLongArrowRight 论文}
%
%\faThumbsUp  \textbf{性格开朗,热爱技术,有较强的自学能力和团队协作能力}
%\end{raggedright}

\section{\faBeer~兴趣}

\tikzset{
    cercle/.pic={
      \node [draw, thick, circle, minimum width=10pt] {\tikzpictext};
    },
  }

%\vspace*{2mm}
\begin{minipage}{\linewidth}
  \begin{tikzpicture}
 	\hspace*{20mm}
  	\pic [pic text={\LARGE \faQq}] {cercle};
    \node[draw=none] at (0,-1.1) {交友};
    \pic [pic text={\Huge \faPaw}] at (20mm,0) {cercle};
    \node[draw=none] at (2,-1.1) {探索};
    \pic [pic text={\Huge \faGooglePlusSquare}] at (40mm,0) {cercle};
    \node[draw=none] at (4,-1.1) {学术};
    \pic [pic text={\Huge \faCamera}] at (60mm,0) {cercle};
    \node[draw=none] at (6,-1.1) {摄影};
    \pic [pic text={\huge \faFileWordO}] at (80mm,0) {cercle};
    \node[draw=none] at (8,-1.1) {Office};
    \pic [pic text={\Huge \faDribbble}] at (100mm,0) {cercle};
    \node[draw=none] at (10,-1.1) {运动};
    \pic [pic text={\LARGE \faYoutubeSquare}] at (120mm,0) {cercle};
    \node[draw=none] at (12,-1.1) {电影};
    \pic [pic text={\LARGE  \faFileCodeO}] at (140mm,0) {cercle};
    \node[draw=none] at (14,-1.1) {编程};
    \pic [pic text={\Huge \faGithub}] at (160mm,0) {cercle};
    \node[draw=none] at (16,-1.1) {开源};
  \end{tikzpicture}
\end{minipage}



\end{document}
